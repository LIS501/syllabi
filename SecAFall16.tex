\documentclass[]{article}
\usepackage{lmodern}
\usepackage{amssymb,amsmath}
\usepackage{ifxetex,ifluatex}
\usepackage{fixltx2e} % provides \textsubscript
\ifnum 0\ifxetex 1\fi\ifluatex 1\fi=0 % if pdftex
  \usepackage[T1]{fontenc}
  \usepackage[utf8]{inputenc}
\else % if luatex or xelatex
  \ifxetex
    \usepackage{mathspec}
  \else
    \usepackage{fontspec}
  \fi
  \defaultfontfeatures{Ligatures=TeX,Scale=MatchLowercase}
\fi
% use upquote if available, for straight quotes in verbatim environments
\IfFileExists{upquote.sty}{\usepackage{upquote}}{}
% use microtype if available
\IfFileExists{microtype.sty}{%
\usepackage{microtype}
\UseMicrotypeSet[protrusion]{basicmath} % disable protrusion for tt fonts
}{}
\usepackage{hyperref}
\hypersetup{unicode=true,
            pdftitle={Information Organization and Access},
            pdfauthor={University of Illinois School of Information Sciences},
            pdfborder={0 0 0},
            breaklinks=true}
\urlstyle{same}  % don't use monospace font for urls
\IfFileExists{parskip.sty}{%
\usepackage{parskip}
}{% else
\setlength{\parindent}{0pt}
\setlength{\parskip}{6pt plus 2pt minus 1pt}
}
\setlength{\emergencystretch}{3em}  % prevent overfull lines
\providecommand{\tightlist}{%
  \setlength{\itemsep}{0pt}\setlength{\parskip}{0pt}}
\setcounter{secnumdepth}{0}
% Redefines (sub)paragraphs to behave more like sections
\ifx\paragraph\undefined\else
\let\oldparagraph\paragraph
\renewcommand{\paragraph}[1]{\oldparagraph{#1}\mbox{}}
\fi
\ifx\subparagraph\undefined\else
\let\oldsubparagraph\subparagraph
\renewcommand{\subparagraph}[1]{\oldsubparagraph{#1}\mbox{}}
\fi

\title{Information Organization and Access}
\author{University of Illinois School of Information Sciences}
\date{Thu, Aug 4, 2016 9:56:00 AM}

\begin{document}
\maketitle

LIS 501A\\
Fall 2016\\
Tuesday 1:00-2:50pm Room 126\\
Thursday 1:00-3:50pm Room 46\\
4 credit hours

Instructor: Maria Bonn\\
Email: mbonn@illinois.edu\\
Office: LIS 338\\
Office Hour: TBA\\
Phone: (734) 417-6937

Instructor: David Dubin\\
Email: ddubin@illinois.edu\\
Office: LIS 330\\
Office Hour: Wednesdays, 2-5pm and by appointment\\
Phone: (217) 244-3275

Instructor: Jodi Schneider\\
Email: jschne23@illinois.edu\\
Office: LIS 334\\
Office Hour: TBA\\
Phone: TBD

\section{Course Description}\label{course-description}

This course provides an intensive and thorough introduction to
fundamentals of information organization and access from the point of
view of the field of library and information science. The course is not
an introduction to LIS as a whole or to the profession of
librarianship--the focus is squarely on information organization and
access. 501 aims to acquaint you with the principal problems of
information organization and access, the main streams of thought, and
the key thinkers and contributors. The material covered is broad in
scope and applicable to a wide variety of settings and systems. The
course emphasizes the central position of people, communities, and
information users in problems of information organization and access.

\section{Course Overview}\label{course-overview}

The central themes of the course are:

\begin{enumerate}
\def\labelenumi{\arabic{enumi}.}
\tightlist
\item
  Who uses information, how they use it, and what constraints shape
  their use of information;
\item
  How recorded knowledge can be organized and structured;\\
\item
  Ways of providing access to the world's knowledge.
\end{enumerate}

\section{Learning Objectives}\label{learning-objectives}

\begin{enumerate}
\def\labelenumi{\arabic{enumi}.}
\tightlist
\item
  To provide a foundation for further study in library and information
  science, including an appreciation for forms of systematic research in
  LIS.
\item
  To introduce central concepts, theories, principles, research issues,
  and people associated with the practice and study of information
  organization and access.
\item
  To advance a common set of ideas that help to define the profession's
  orientation toward problems of information organization and access.
\end{enumerate}

\section{Course Materials}\label{course-materials}

There is no required textbook for this course; required articles and
book chapters are available electronically, through the UIUC Library's
electronic reserves system
(\url{https://reserves.library.illinois.edu/ares/}), the UIUC Library's
e-journals search tool
(\url{http://openurl.library.uiuc.edu/sfxlcl3/az}), or the open Web (as
indicated).

The textbook \emph{The Discipline of Organizing} by Robert J. Glushko is
recommended for those new to the field. It provides a discussion of many
of the concepts we will discuss in class, and can be accessed online at
\url{http://vufind.carli.illinois.edu/vf-uiu/Record/uiu_7572272}.

This course will introduce you to many terms and concepts that may be
new to you.\\
Remember that the Library provides access to some reference sources
online that may be helpful in providing additional background and
explanations.\\
These titles are linked from:
\url{http://www.library.illinois.edu/lsx/findit/tools/encyclopedias.html}
and
\url{http://www.library.illinois.edu/lsx/findit/tools/dictionaries.html}.\\
They include: \emph{ODLIS: Online Dictionary for Library and Information
Science} and the \emph{Encyclopedia of Library and Information
Sciences}.

\section{About Maria Bonn}\label{about-maria-bonn}

Maria is a senior lecturer at the Graduate School of Library and
Information Science at the GSLIS. She is developing courses on the role
of libraries in scholarly communication and publishing. Prior to her
teaching appointment, Maria served as the associate university librarian
for publishing at the University of Michigan Library, with
responsibility for publishing and scholarly communications initiatives,
including the University of Michigan Press, the Library's Scholarly
Publishing Office, the institutional repository (Deep Blue), the
Copyright Office, and the Text Creation Partnership. She has also been
an assistant professor of English at Albion College and taught at
Sichuan International Studies University (Chongqing, China) and Bilkent
University (Ankara, Turkey). She received a bachelor's degree with a
major in English from the University of Rochester, masters and doctoral
degrees in American Literature from SUNY Buffalo, and a masters in
information and library science from the University of Michigan.

\section{About Dave Dubin}\label{about-dave-dubin}

David Dubin is a Research Associate Professor at GSLIS. His research
explores the foundations of information representation and description
as well as issues of expression and encoding in documents and digital
information resources.

\section{About Jodi Schneider}\label{about-jodi-schneider}

Jodi Schneider is an assistant professor at the School of Information
Sciences. She studies scholarly communication and social media through
the lens of arguments, evidence, and persuasion. She is developing
Linked Data (ontologies, metadata, Semantic Web) approaches to manage
scientific evidence. Jodi holds degrees in informatics (Ph.D., National
University of Ireland, Galway), library \& information science (M.S.
UIUC), mathematics (M.A.~UT-Austin), and liberal arts (B.A., Great
Books, St.~John's College). She worked in academic libraries and
bookstores for 6 years. She has also held research positions across the
U.S. as well as in Ireland, England, France, and Chile.

\section{Library Resources}\label{library-resources}

\url{http://www.library.illinois.edu/lsx/}\\
lislib@library.illinois.edu\\
Phone: 217-333-3804

\section{Writing Resources}\label{writing-resources}

The Writers Workshop provides free consultations. For more information
see \url{http://www.cws.illinois.edu/workshop/}

\section{Academic Integrity}\label{academic-integrity}

Please review and reflect on the academic integrity policy of the
University of Illinois,
\url{http://admin.illinois.edu/policy/code/article1_part4_1-401.html} to
which we subscribe. By turning in materials for review, you certify that
all work presented is your own and has been done by you independently,
or as a member of a designated group for group assignments. If, in the
course of your writing, you use the words or ideas of another writer,
proper acknowledgement must be given (using APA, Chicago, or MLA style).
Not to do so is to commit plagiarism, a form of academic dishonesty. If
you are not absolutely clear on what constitutes plagiarism and how to
cite sources appropriately, now is the time to learn. Please ask me!
Please be aware that the consequences for plagiarism or other forms of
academic dishonesty will be severe. Students who violate university
standards of academic integrity are subject to disciplinary action,
including a reduced grade, failure in the course, and suspension or
dismissal from the University.

\section{Statement of Inclusion}\label{statement-of-inclusion}

\url{http://www.inclusiveillinois.illinois.edu/chancellordivstmtswf.html\#ValueStmt}

As the state's premier public university, the University of Illinois at
Urbana-Champaign's core mission is to serve the interests of the diverse
people of the state of Illinois and beyond. The institution thus values
inclusion and a pluralistic learning and research environment, one which
we respect the varied perspectives and lived experiences of a diverse
community and global workforce. We support diversity of worldviews,
histories, and cultural knowledge across a range of social groups
including race, ethnicity, gender identity, sexual orientation,
abilities, economic class, religion, and their intersections.

\section{Accessibility Statement}\label{accessibility-statement}

To obtain accessibility-related academic adjustments and/or auxiliary
aids, students with disabilities must contact the course instructor and
the Disability Resources and Educational Services (DRES) as soon as
possible. To contact DRES you may visit 1207 S. Oak St., Champaign, call
333-4603 (V/TTY), or e-mail a message to disability@uiuc.edu.

\section{Assignments and Evaluation}\label{assignments-and-evaluation}

All assignments are required for all students. Completing all
assignments is not a guarantee of a passing grade. All work must be
completed in order to pass this class. Late or incomplete assignments
will not be given full credit unless the student has contacted the
instructor prior to the due date of the assignment (or in the case of
emergencies, as soon as practicable). There will be three main
assignments, and ten labs. The weighting of each assignment in the final
grade is noted below.

Assignments, Exercises \& Grade Distribution:

\begin{itemize}
\tightlist
\item
  Assignment 1: Information Needs/Information Seeking Behavior. Due
  September due date at 11:59 PM (20\%).
\item
  Assignment 2: Digital Collections Assessment. Due October due date at
  11:59 PM (20\%).
\item
  Assignment 3: Final Project. Due October due date at 11:59 PM (40\%).
\item
  10 Labs for Attendance and Completion/Class Participation (20\%)
\end{itemize}

Labs:

\begin{enumerate}
\def\labelenumi{\arabic{enumi}.}
\tightlist
\item
  Library resources (August 25)
\item
  ITD on computing resources (September 1)
\item
  Zotero bibliography and formatting (September 29)
\item
  Cultural heritage object description (September 15)
\item
  Research methods exercise (September 22)
\item
  Pandoc encoding and transformation (September 8)
\item
  Bibo/DC output from Zotero (October 6)
\item
  Collection stewardship exercise (October 13)
\item
  Poster prep (October 20)
\item
  Poster presentations (October 27)
\end{enumerate}

Grading Scale:

94-100 = A\\
90-93 = A-\\
87-89 = B+\\
83-86 = B\\
80-82 = B-\\
77-79 = C+\\
73-76 = C\\
70-72 = C-\\
67-69 = D+\\
63-66 = D\\
60-62 = D-\\
59 and below = F

\section{Assignment 1 Information Needs/Information Seeking
Behavior}\label{Asgt1}

Due September due date at 11:59 PM. This assignment is worth 20 points.

\subsection{Rationale}\label{rationale}

The purpose of this exercise is two-fold. First, it is intended to allow
you to examine the factors that condition the information needs or
influence the information seeking behavior of a specific user group.
Second, it affords you the opportunity to explore what types of research
methods are used to do research on concrete user groups. The handout
from the first lab of class will assist you with this exercise. Methods
Handout If you'd like to know more about the method(s) in your article,
you can refer to this Research Methods portal:
\url{http://InformationR.net/rm/}

\subsection{Tasks}\label{tasks}

\begin{enumerate}
\def\labelenumi{\arabic{enumi}.}
\tightlist
\item
  Select a category of information user that interests you (e. g., high
  school student, scientist, health care consumer, migrant farm worker).
\item
  Identify a published research study that investigates information
  needs and/or information seeking behavior of individuals from this
  population.
\end{enumerate}

\subsection{Deliverable}\label{deliverable}

In a brief essay of 600-700 words (1-2 pages) discuss your article's
research question, methods, findings, and implications for design of
information systems and services (in-person, web-based, or both).
Integrate relevant readings from those assigned for class and especially
those you have read to date (and from the users and information needs
session in particular) in support of your arguments.

\subsection{Submitting}\label{submitting}

\begin{enumerate}
\def\labelenumi{\arabic{enumi}.}
\tightlist
\item
  Upload your essay to the Assignment 1 Moodle dropbox.
\item
  Post a one-paragraph summary of your article on the Assignment 1
  discussion forum.
\item
  Bring a copy of your article to class for small group discussion.
\end{enumerate}

\subsection{Strategies for locating research
studies}\label{strategies-for-locating-research-studies}

Articles indexed under the subject ``Information needs'' or
``Information-seeking behavior'' in Library \& Information Science
Source (link from \url{http://www.library.illinois.edu/lsx/}
\url{http://openurl.library.uiuc.edu/sfxlcl3/az)}

Articles indexed under the descriptor ``Information seeking behaviour''
or ``User needs'' in LISA (link from
\url{http://www.library.illinois.edu/lsx/}
\url{http://openurl.library.uiuc.edu/sfxlcl3/az}
\url{http://www.library.uiuc.edu/orr/results.php?types=A\&subject=29)}

Studies need not be recently published, but an example of a recent study
of this type is the following:

\begin{quote}
Darby, P.; Clough, P. (2013). Investigating the information-seeking
behaviour of genealogists and family historians. Journal of Information
Science 39(1): 73-84.
\end{quote}

\begin{quote}
People are increasingly investigating their family history (or
genealogy) as part of their everyday information-seeking activities.
This paper provides insight into this behaviour and presents a new
conceptual model that captures the stages of activity carried out during
people's lifelong family history research. The model offers a
multi-phase view of the research process, intended to illustrate: (a)
the different research phases themselves; (b) the inter-relationship
between phases; (c) distinct phase-specific behaviours; and (d)
phase-specific resource preferences. Data collected from amateur family
historians by interview and questionnaire has helped to validate the
model and provide insights into the information resources used. The
findings complement existing knowledge about family history research and
will benefit: family historians as they seek to navigate within the
research process; providers of genealogical resources as they seek to
better support users; and academics as they study information-seeking
behaviours in various contexts.
\end{quote}

To determine whether the full text of an article is available online,
search for the journal title at:
\url{http://openurl.library.uiuc.edu/sfxlcl3/az}. The text of the
article may also be freely available on the web (e. g., on the author's
web site), so you might also try doing a Google search using the article
title. If the text is not available online, you will need to request a
copy of the article. See the LibGuide for Online and Continuing
Education Student Resources from the Library:

\url{http://uiuc.libguides.com/content.php?pid=28713\&sid=209698}
\url{http://uiuc.libguides.com/content.php?pid=28713\&sid=214766}
\url{http://uiuc.libguides.com/content.php?pid=28713\&sid=214766}

\section{Topic Schedule}\label{topic-schedule}

Week 1, August 23: LIS Education and professional life

\begin{itemize}
\tightlist
\item
  Background: (Glushko
  \protect\hyperlink{ref-glushkoux5ffoundationux5f2015}{2015}\protect\hyperlink{ref-glushkoux5ffoundationux5f2015}{c}),
  (Wright
  \protect\hyperlink{ref-wrightux5fsecretux5f2014}{2014}\protect\hyperlink{ref-wrightux5fsecretux5f2014}{b}),
  (Wright
  \protect\hyperlink{ref-wrightux5fcatalogingux5f2014}{2014}\protect\hyperlink{ref-wrightux5fcatalogingux5f2014}{a}),
  (Levie and Sofidoc Productions.
  \protect\hyperlink{ref-levieux5fmanux5f2004}{2004}).
\item
  Required: (Lavoie, Dempsey, and Connaway
  \protect\hyperlink{ref-lavoieux5fmakingux5f2006}{2006}), (Dyson
  \protect\hyperlink{ref-dysonux5fhowux5f2011}{2011}), (Bush
  \protect\hyperlink{ref-bushux5fasux5f1945}{1945}), (Buckland
  \protect\hyperlink{ref-bucklandux5fwhatux5f1997}{1997}), (Bates
  \protect\hyperlink{ref-batesux5finvisibleux5f1999}{1999}).
\end{itemize}

Week 2, August 30: Users and information needs

\begin{itemize}
\tightlist
\item
  Background: (Glushko
  \protect\hyperlink{ref-glushkoux5finteractionsux5f2015}{2015}\protect\hyperlink{ref-glushkoux5finteractionsux5f2015}{d}),
  (Naumer and Fisher
  \protect\hyperlink{ref-naumerux5finformationux5f2009}{2009}), (Wilson
  \protect\hyperlink{ref-wilsonux5finformationux5f2008}{2008}), (Miksa
  \protect\hyperlink{ref-miksaux5finformationux5f2009}{2009}), (Foss et
  al. \protect\hyperlink{ref-fossux5fchildrensux5f2012}{2012}), (Hyder
  \protect\hyperlink{ref-hyderux5freadingux5f2014}{2014}), (David
  Johnson
  \protect\hyperlink{ref-davidux5fjohnsonux5fhealth-relatedux5f2014}{2014}),
  (Connaway and Faniel
  \protect\hyperlink{ref-connawayux5freorderingux5f2014}{2014}), (Marcia
  J. Bates
  \protect\hyperlink{ref-marciaux5fj.ux5fbatesux5finformationux5f2009}{2009}),
  (Connaway and Powell
  \protect\hyperlink{ref-connawayux5fselectingux5f2010}{2010}).
\item
  Required: (Connaway, Dickey, and Radford
  \protect\hyperlink{ref-connawayux5fifux5f2011}{2011}), (Blair
  \protect\hyperlink{ref-blairux5freadingux5f2003}{2003}), (Bawden and
  Robinson \protect\hyperlink{ref-bawdenux5fdarkux5f2009}{2009}).
\end{itemize}

Week 3, September 6: Research Methods

\begin{itemize}
\tightlist
\item
  Background: (Fidel
  \protect\hyperlink{ref-fidelux5fareux5f2008}{2008}), (Brett Sutton
  \protect\hyperlink{ref-brettux5fsuttonux5fqualitativeux5f2009}{2009}),
  (Sandstrom and Sandstrom
  \protect\hyperlink{ref-sandstromux5fuseux5f1995}{1995}), (Connaway and
  Powell \protect\hyperlink{ref-connawayux5fselectingux5f2010}{2010}).
\item
  Required: (Shachaf and Horowitz
  \protect\hyperlink{ref-shachafux5fareux5f2006}{2006}), (Whitmire
  \protect\hyperlink{ref-whitmireux5fracialux5f1999}{1999}).
\end{itemize}

Week 4, September 13: Structures and Standards

\begin{itemize}
\tightlist
\item
  Background: (Glushko
  \protect\hyperlink{ref-glushkoux5fdescribingux5f2015}{2015}\protect\hyperlink{ref-glushkoux5fdescribingux5f2015}{b}).
\item
  Required: (Vogt
  \protect\hyperlink{ref-vogtux5fescienceux5f2013}{2013}), (DeRose
  \protect\hyperlink{ref-deroseux5fwhatux5f2014}{2014}), (Bettels and
  Bishop \protect\hyperlink{ref-bettelsux5funicode:ux5f1993}{1993}),
  (Coombs, Renear, and DeRose
  \protect\hyperlink{ref-coombsux5fmarkupux5f1987}{1987}).
\end{itemize}

Week 5, September 20: Approaches to organizing information

\begin{itemize}
\tightlist
\item
  Background: (Glushko
  \protect\hyperlink{ref-glushkoux5fresourcesux5f2015}{2015}\protect\hyperlink{ref-glushkoux5fresourcesux5f2015}{e}).
\item
  Required: (Kennedy
  \protect\hyperlink{ref-kennedyux5fnineux5f2008}{2008}), (Maxwell
  \protect\hyperlink{ref-maxwellux5fbibliographicux5f2010}{2010}),
  (Warren \protect\hyperlink{ref-warrenux5f2015ux5f2015}{2015}), (Swoger
  \protect\hyperlink{ref-swogerux5f.ux5f2012}{2012}).
\end{itemize}

Week 6, September 27: Collections

\begin{itemize}
\tightlist
\item
  Background: (Junus
  \protect\hyperlink{ref-junusux5fdigitalux5f2014}{2014}), (Glushko
  \protect\hyperlink{ref-glushkoux5factivitiesux5f2015}{2015}\protect\hyperlink{ref-glushkoux5factivitiesux5f2015}{a}).
\item
  Required: (Hadro
  \protect\hyperlink{ref-hadroux5fwhatsux5f2013}{2013}), (``Update on
  the Twitter Archive at the Library of Congress.''
  \protect\hyperlink{ref-ux5fupdateux5f2013}{2013}), (Hunter and
  Oehlerts \protect\hyperlink{ref-hunterux5ftwoux5f1981}{1981}), (Lewis
  \protect\hyperlink{ref-lewisux5fstacksux5f2013}{2013}).
\end{itemize}

Week 7, October 4: Preservation

\begin{itemize}
\tightlist
\item
  Background: (Glushko
  \protect\hyperlink{ref-glushkoux5forganizingux5f2015}{2015}\protect\hyperlink{ref-glushkoux5forganizingux5f2015}{f}).
\item
  Required: (Shilton and Srinivasan
  \protect\hyperlink{ref-shiltonux5fparticipatoryux5f2007}{2007}),
  (Teper \protect\hyperlink{ref-teperux5fselectionux5f2014}{2014}).
\end{itemize}

Week 8, October 11: Search and discovery

\begin{itemize}
\tightlist
\item
  Background: (Gossen and Nürnberger
  \protect\hyperlink{ref-gossenux5fspecificsux5f2013}{2013}), (Hearst
  \protect\hyperlink{ref-hearstux5fevaluationux5f2009}{2009}), (Duffy
  \protect\hyperlink{ref-duffyux5fsearchingux5f2013}{2013}), (Bates
  \protect\hyperlink{ref-batesux5fwhatux5f2007}{2007}).
\item
  Required: (Saarinen and Vakkari
  \protect\hyperlink{ref-saarinenux5fsignux5f2013}{2013}), (Bawden
  \protect\hyperlink{ref-bawdenux5fencounteringux5f2011}{2011}), (Barton
  and Mak \protect\hyperlink{ref-bartonux5foldux5f2012}{2012}), (Adkins
  and Bossaller
  \protect\hyperlink{ref-adkinsux5ffictionux5f2007}{2007}).
\end{itemize}

Week 9, October 18: Evaluation of systems and services

\begin{itemize}
\tightlist
\item
  Background.
\item
  Required: (Asher, Duke, and Wilson
  \protect\hyperlink{ref-asherux5fpathsux5f2013}{2013}), (Voorhees
  \protect\hyperlink{ref-petersux5fphilosophyux5f2002}{2002}).
\end{itemize}

Week 10, October 25: Subject analysis and subject languages

\begin{itemize}
\tightlist
\item
  Background: (Kreyche
  \protect\hyperlink{ref-kreycheux5fsubjectux5f2013}{2013}), (Mitchell
  and Vizine-Goetz
  \protect\hyperlink{ref-mitchellux5fdeweyux5f2009}{2009}), (Chan and
  Hodges \protect\hyperlink{ref-chanux5flibraryux5f2009}{2009}),
  (Anderson and Pérez-Carballo
  \protect\hyperlink{ref-andersonux5flibraryux5f2009}{2009}), (Beghtol
  \protect\hyperlink{ref-beghtolux5fclassificationux5f2009}{2009}).
\item
  Required: (Lee
  \protect\hyperlink{ref-leeux5findigenousux5f2011}{2011}), (Higgins
  \protect\hyperlink{ref-higginsux5flibraryux5f2012}{2012}), (Fister
  \protect\hyperlink{ref-fisterux5fdeweyux5f2009}{2009}), (Buckland
  \protect\hyperlink{ref-bucklandux5fobsolescenceux5f2012}{2012}),
  (Brown-Sica and Beall
  \protect\hyperlink{ref-brown-sicaux5flibraryux5f2008}{2008}).
\end{itemize}

\section*{References}\label{references}
\addcontentsline{toc}{section}{References}

\hypertarget{refs}{}
\hypertarget{ref-adkinsux5ffictionux5f2007}{}
Adkins, Denice, and Jenny E. Bossaller. 2007. ``Fiction Access Points
Across Computer-Mediated Book Information Sources: A Comparison of
Online Bookstores, Reader Advisory Databases, and Public Library
Catalogs.'' \emph{Library \& Information Science Research} 29 (3):
354--68.
doi:\href{https://doi.org/10.1016/j.lisr.2007.03.004}{10.1016/j.lisr.2007.03.004}.

\hypertarget{ref-andersonux5flibraryux5f2009}{}
Anderson, James Doig, and José Pérez-Carballo. 2009. ``Library of
Congress Subject Headings (LCSH).'' In \emph{Encyclopedia of Library and
Information Sciences, Third Edition}, 3392--3405. Taylor \& Francis.
\url{http://www.tandfonline.com/doi/abs/10.1081/E-ELIS3-120043717}.

\hypertarget{ref-asherux5fpathsux5f2013}{}
Asher, Andrew D, Lynda M Duke, and Suzanne Wilson. 2013. ``Paths of
Discovery: Comparing the Search Effectiveness of EBSCO Discovery
Service, Summon, Google Scholar, and Conventional Library Resources.''
\emph{College \& Research Libraries} 74 (5): 464--88.
doi:\href{https://doi.org/10.5860/crl-374}{10.5860/crl-374}.

\hypertarget{ref-bartonux5foldux5f2012}{}
Barton, Joshua, and Lucas Mak. 2012. ``Old Hopes, New Possibilities:
Next-Generation Catalogues and the Centralization of Access.''
\emph{Library Trends} 61 (1): 83--106.
doi:\href{https://doi.org/10.1353/lib.2012.0030}{10.1353/lib.2012.0030}.

\hypertarget{ref-batesux5finvisibleux5f1999}{}
Bates, Marcia J. 1999. ``The Invisible Substrate of Information
Science.'' \emph{Journal of the American Society for Information
Science} 50 (12): 1043--50.
doi:\href{https://doi.org/10.1002/(SICI)1097-4571(1999)50:12\%3C1043::AID-ASI1\%3E3.0.CO;2-X}{10.1002/(SICI)1097-4571(1999)50:12\textless{}1043::AID-ASI1\textgreater{}3.0.CO;2-X}.

\hypertarget{ref-batesux5fwhatux5f2007}{}
---------. 2007. ``What Is Browsing---really? A Model Drawing from
Behavioural Science Research.'' \emph{Information Research} 12 (4).
\url{http://www.informationr.net/ir/12-4/paper330.html}.

\hypertarget{ref-bawdenux5fencounteringux5f2011}{}
Bawden, David. 2011. ``Encountering on the Road to Serendip? Browsing in
New Information Environments.'' In \emph{Innovations in Information
Retrieval: Perspectives for Theory and Practice London}. London: Facet
Publishing. \url{https://reserves.library.illinois.edu/}.

\hypertarget{ref-bawdenux5fdarkux5f2009}{}
Bawden, David, and Lyn Robinson. 2009. ``The Dark Side of Information:
Overload, Anxiety and Other Paradoxes and Pathologies.'' \emph{J. Inf.
Sci.} 35 (2): 180--91.
doi:\href{https://doi.org/10.1177/0165551508095781}{10.1177/0165551508095781}.

\hypertarget{ref-beghtolux5fclassificationux5f2009}{}
Beghtol, Clare. 2009. ``Classification Theory.'' In \emph{Encyclopedia
of Library and Information Sciences, Third Edition}, 1045--60. Taylor \&
Francis.
\url{http://www.tandfonline.com/doi/abs/10.1081/E-ELIS3-120043230}.

\hypertarget{ref-bettelsux5funicode:ux5f1993}{}
Bettels, Jürgen, and F. Avery Bishop. 1993. ``Unicode: A Universal
Character Code.'' \emph{Digital Tech. J.} 5 (3): 21--31.
\url{http://www.hpl.hp.com/hpjournal/dtj/vol5num3/vol5num3art2.pdf}.

\hypertarget{ref-blairux5freadingux5f2003}{}
Blair, Ann. 2003. ``Reading Strategies for Coping With Information
Overload ca.1550-1700.'' \emph{Journal of the History of Ideas} 64 (1):
11--28.
doi:\href{https://doi.org/10.1353/jhi.2003.0014}{10.1353/jhi.2003.0014}.

\hypertarget{ref-brettux5fsuttonux5fqualitativeux5f2009}{}
Brett Sutton. 2009. ``Qualitative Research Methods in Library and
Information Science {[}ELIS Classic{]}.'' In \emph{Encyclopedia of
Library and Information Sciences, Third Edition}, null:4380--93. null.
Taylor \& Francis. \url{http://dx.doi.org/10.1081/E-ELIS3-120044785}.

\hypertarget{ref-brown-sicaux5flibraryux5f2008}{}
Brown-Sica, Margaret, and Jeffrey Beall. 2008. ``Library 2.0 and the
Problem of Hate Speech.'' \emph{Electronic Journal of Academic and
Special Librarianship} 9 (2).
\url{http://southernlibrarianship.icaap.org/content/v09n02/brown-sica_m01.html}.

\hypertarget{ref-bucklandux5fwhatux5f1997}{}
Buckland, Michael K. 1997. ``What Is a `Document'?'' \emph{Journal of
the American Society for Information Science} 48 (9): 804--9.
doi:\href{https://doi.org/10.1002/(SICI)1097-4571(199709)48:9\%3C804::AID-ASI5\%3E3.0.CO;2-V}{10.1002/(SICI)1097-4571(199709)48:9\textless{}804::AID-ASI5\textgreater{}3.0.CO;2-V}.

\hypertarget{ref-bucklandux5fobsolescenceux5f2012}{}
---------. 2012. ``Obsolescence in Subject Description.'' \emph{Journal
of Documentation} 68 (2): 154--61.
\url{http://www.emeraldinsight.com/doi/abs/10.1108/00220411211209168}.

\hypertarget{ref-bushux5fasux5f1945}{}
Bush, Vannevar. 1945. ``As We May Think.'' \emph{The Atlantic Monthly}
176 (1): 101--8.
\url{http://www.theatlantic.com/unbound/flashbks/computer/bushf.htm}.

\hypertarget{ref-chanux5flibraryux5f2009}{}
Chan, Lois Mai, and Theodora L. Hodges. 2009. ``Library of Congress
Classification (LCC).'' In \emph{Encyclopedia of Library and Information
Sciences, Third Edition}, 3383--91. Taylor \& Francis.
\url{http://www.tandfonline.com/doi/abs/10.1081/E-ELIS3-120043714}.

\hypertarget{ref-connawayux5freorderingux5f2014}{}
Connaway, Lynn Silipigni, and Ixchel M. Faniel. 2014. \emph{Reordering
Ranganathan: Shifting User Behaviors, Shifting Priorities}. Dublin, OH:
OCLC Research.
\url{http://www.oclc.org/content/dam/research/publications/library/2014/oclcresearch-reordering-ranganathan-2014.pdf}.

\hypertarget{ref-connawayux5fselectingux5f2010}{}
Connaway, Lynn Silipigni, and Ronald R. Powell. 2010. ``Selecting the
Research Method.'' In \emph{Basic Research Methods for Librarians},
71--106. Library and Information Science Text Series. Santa Barbara,
Calif: Libraries Unlimited.
\url{http://search.ebscohost.com/login.aspx?direct=true\&db=nlebk\&AN=348676\&site=ehost-live}.

\hypertarget{ref-connawayux5fifux5f2011}{}
Connaway, Lynn, Timothy Dickey, and Marie Radford. 2011. ```'If It Is
Too Inconvenient I'm Not Going After It:' Convenience as a Critical
Factor in Information-Seeking Behaviors'.'' \emph{Library \& Information
Science Research} 33 (3): 179--90.
doi:\href{https://doi.org/http://dx.doi.org/10.1016/j.lisr.2010.12.002}{http://dx.doi.org/10.1016/j.lisr.2010.12.002}.

\hypertarget{ref-coombsux5fmarkupux5f1987}{}
Coombs, James H., Allen H. Renear, and Steven J. DeRose. 1987. ``Markup
Systems and the Future of Scholarly Text Processing.'' \emph{Commun.
ACM} 30 (11): 933--47.
doi:\href{https://doi.org/10.1145/32206.32209}{10.1145/32206.32209}.

\hypertarget{ref-davidux5fjohnsonux5fhealth-relatedux5f2014}{}
David Johnson, J. 2014. ``Health-Related Information Seeking: Is It
Worth It?'' \emph{Information Processing \& Management} 50 (5): 708--17.
doi:\href{https://doi.org/10.1016/j.ipm.2014.06.001}{10.1016/j.ipm.2014.06.001}.

\hypertarget{ref-deroseux5fwhatux5f2014}{}
DeRose, Steven J. 2014. ``What Do We Still Lack? Or: Prolegomena to Any
Future Hypertext System.'' In \emph{Proceedings of the Symposium on
HTML5 and XML}. Vol. 14. Balisage Series on Markup Technologies.
Washington, DC: Muberry Technologies, Inc.
doi:\href{https://doi.org/10.4242/BalisageVol14.DeRose01}{10.4242/BalisageVol14.DeRose01}.

\hypertarget{ref-duffyux5fsearchingux5f2013}{}
Duffy, Eamon P. 2013. ``Searching HathiTrust: Old Concepts in a New
Context.'' \emph{Partnership: The Canadian Journal of Library and
Information Practice and Research} 8 (1).
\url{https://journal.lib.uoguelph.ca/index.php/perj/article/view/2503}.

\hypertarget{ref-dysonux5fhowux5f2011}{}
Dyson, Freeman. 2011. ``How We Know.'' \emph{The New York Review of
Books}, March.
\url{http://www.nybooks.com/articles/archives/2011/mar/10/how-we-know/}.

\hypertarget{ref-fidelux5fareux5f2008}{}
Fidel, Raya. 2008. ``Are We There yet?: Mixed Methods Research in
Library and Information Science.'' \emph{Library \& Information Science
Research} 30 (4): 265--72.
doi:\href{https://doi.org/10.1016/j.lisr.2008.04.001}{10.1016/j.lisr.2008.04.001}.

\hypertarget{ref-fisterux5fdeweyux5f2009}{}
Fister, Barbara. 2009. ``The Dewey Dilemma.'' \emph{Library Journal} 134
(16): 22--25. \url{http://eric.ed.gov/?id=EJ859403}.

\hypertarget{ref-fossux5fchildrensux5f2012}{}
Foss, Elizabeth, Allison Druin, Robin Brewer, Phillip Lo, Luis Sanchez,
Evan Golub, and Hilary Hutchinson. 2012. ``Children's Search Roles at
Home: Implications for Designers, Researchers, Educators, and Parents.''
\emph{Journal of the American Society for Information Science and
Technology} 63 (3): 558--73.
doi:\href{https://doi.org/10.1002/asi.21700}{10.1002/asi.21700}.

\hypertarget{ref-glushkoux5factivitiesux5f2015}{}
Glushko, Robert J. 2015a. ``Activities in Organizing Systems.'' In
\emph{The Discipline of Organizing}, 3rd ed., 97--168. O'Reilly.
\url{http://disciplineoforganizing.org/}.

\hypertarget{ref-glushkoux5fdescribingux5f2015}{}
---------. 2015b. ``Describing Relationships and Structures.'' In
\emph{The Discipline of Organizing}, 3rd ed., 295--344. O'Reilly.
\url{http://disciplineoforganizing.org/}.

\hypertarget{ref-glushkoux5ffoundationux5f2015}{}
---------. 2015c. ``Foundation for Organizing Systems.'' In \emph{The
Discipline of Organizing}, 3rd ed., 33--96. O'Reilly.
\url{http://disciplineoforganizing.org/}.

\hypertarget{ref-glushkoux5finteractionsux5f2015}{}
---------. 2015d. ``Interactions with Resources.'' In \emph{The
Discipline of Organizing}, 3rd ed., 499--542. O'Reilly.
\url{http://disciplineoforganizing.org/}.

\hypertarget{ref-glushkoux5fresourcesux5f2015}{}
---------. 2015e. ``Resources in Organizing Systems.'' In \emph{The
Discipline of Organizing}, 3rd ed., 169--230. O'Reilly.
\url{http://disciplineoforganizing.org/}.

\hypertarget{ref-glushkoux5forganizingux5f2015}{}
---------. 2015f. ``The Organizing System: Roadmap.'' In \emph{The
Discipline of Organizing}, 3rd ed., 543--70. O'Reilly.
\url{http://disciplineoforganizing.org/}.

\hypertarget{ref-gossenux5fspecificsux5f2013}{}
Gossen, Tatiana, and Andreas Nürnberger. 2013. ``Specifics of
Information Retrieval for Young Users: A Survey.'' \emph{Information
Processing \& Management} 49 (4): 739--56.
doi:\href{https://doi.org/10.1016/j.ipm.2012.12.006}{10.1016/j.ipm.2012.12.006}.

\hypertarget{ref-hadroux5fwhatsux5f2013}{}
Hadro, J. 2013. ``What's the Problem with Self-Publishing?''
\emph{Library Journa} 138 (7): 34--36.
\href{http://lj.libraryjournal.com/2013/04/publishing/whats-the-\%20problem-with-self-publishing/}{http://lj.libraryjournal.com/2013/04/publishing/whats-the- problem-with-self-publishing/}.

\hypertarget{ref-hearstux5fevaluationux5f2009}{}
Hearst, Marti A. 2009. ``The Evaluation of Search User Interfaces.'' In
\emph{Search User Interfaces}. Cambridge: Cambridge University Press.
\url{http://searchuserinterfaces.com/book/sui_ch2_evaluation.html}.

\hypertarget{ref-higginsux5flibraryux5f2012}{}
Higgins, Colin. 2012. ``Library of Congress Classification: Teddy
Roosevelt's World in Numbers?'' \emph{Cataloging \& Classification
Quarterly} 50 (4): 249--62.
\url{http://www.tandfonline.com/doi/abs/10.1080/01639374.2012.658989}.

\hypertarget{ref-hunterux5ftwoux5f1981}{}
Hunter, N. C., Legg, and J.B. Oehlerts. 1981. ``Two Librarians, an
Archivist, and 13,000 Images: Collaborating to Build a Digital
Collection.'' \emph{Library Quarterly} 80 (1): 81--103.
doi:\href{https://doi.org/10.1086/648464}{10.1086/648464}.

\hypertarget{ref-hyderux5freadingux5f2014}{}
Hyder, Eileen Mary. 2014. ``Reading Groups and Social Justice.'' In
\emph{Reading Groups, Libraries and Social Inclusion: Experiences of
Blind and Partially Sighted People}, 49--63. Farnham, Surrey, England :
Burlington, VT: Ashgate Publishing, Ltd.

\hypertarget{ref-junusux5fdigitalux5f2014}{}
Junus, Ranti. 2014. ``Digital Collections and Accessibility.'' MSU
Libraries Blogs. \emph{Digital Scholarship Collaborative Sandbox}.
\url{http://blogpublic.lib.msu.edu/index.php/dscsandbox/digital-collection-and-accessibility}.

\hypertarget{ref-kennedyux5fnineux5f2008}{}
Kennedy, M.R. 2008. ``Nine Questions to Guide You in Choosing a Metadata
Schema.'' \emph{Journal of Digital Information} 9 (1).
\url{http://journals.tdl.org/jodi/article/view/226/205}.

\hypertarget{ref-kreycheux5fsubjectux5f2013}{}
Kreyche, Michael. 2013. ``Subject Headings in Spanish: The Lcsh-Es. Org
Bilingual Database.'' \emph{Cataloging \& Classification Quarterly} 51
(4): 389--403.
\url{http://www.tandfonline.com/doi/abs/10.1080/01639374.2012.740610}.

\hypertarget{ref-lavoieux5fmakingux5f2006}{}
Lavoie, Brian, Lorcan Dempsey, and Lynn Silipigni Connaway. 2006.
``Making Data Work Harder.'' \emph{Library Journal} 131 (1): 40--42.
\url{http://search.ebscohost.com/login.aspx?direct=true\&db=a9h\&AN=19426604\&site=ehost-live}.

\hypertarget{ref-leeux5findigenousux5f2011}{}
Lee, Deborah. 2011. ``Indigenous Knowledge Organization: A Study of
Concepts, Terminology, Structure and (Mostly) Indigenous Voices.''
\emph{Partnership: The Canadian Journal of Library and Information
Practice and Research} 6 (1).
\url{https://journal.lib.uoguelph.ca/index.php/perj/article/view/1427}.

\hypertarget{ref-levieux5fmanux5f2004}{}
Levie, Françoise., and Sofidoc Productions. 2004. ``The Man Who Wanted
to Classify the World.''
\url{http://www.aspresolver.com/aspresolver.asp?VASC;1641522}.

\hypertarget{ref-lewisux5fstacksux5f2013}{}
Lewis, D. W. 2013. ``From Stacks to the Web: The Transformation of
Academic Library Collecting.'' \emph{College \& Research Libraries} 14
(2): 159--76.
doi:\href{https://doi.org/10.5860/crl-309}{10.5860/crl-309}.

\hypertarget{ref-marciaux5fj.ux5fbatesux5finformationux5f2009}{}
Marcia J. Bates. 2009. ``Information.'' In \emph{Encyclopedia of Library
and Information Sciences, Third Edition}, null:2347--60. null. Taylor \&
Francis. \url{http://dx.doi.org/10.1081/E-ELIS3-120045519}.

\hypertarget{ref-maxwellux5fbibliographicux5f2010}{}
Maxwell, R.L. 2010. ``Bibliographic Control. 497-505.''
\emph{Encyclopedia of Library and Information Sciences,}
\href{locate\%20from:\%20http://www.library.illinois.edu/lsx/findit/tools/encyclopedias.html\%5D}{locate from: http://www.library.illinois.edu/lsx/findit/tools/encyclopedias.html{]}}.

\hypertarget{ref-miksaux5finformationux5f2009}{}
Miksa, Francis. 2009. ``Information Organization and the Mysterious
Information User.'' \emph{Libraries \& the Cultural Record} 44 (3):
343--70. \url{http://www.jstor.org/stable/25549558}.

\hypertarget{ref-mitchellux5fdeweyux5f2009}{}
Mitchell, Joan S., and Diane Vizine-Goetz. 2009. ``Dewey Decimal
Classification (DDC).'' In \emph{Encyclopedia of Library and Information
Sciences, Third Edition}, 1507--17. Taylor \& Francis.
\url{http://www.tandfonline.com/doi/abs/10.1081/E-ELIS3-120043240}.

\hypertarget{ref-naumerux5finformationux5f2009}{}
Naumer, Charles M., and Karen E. Fisher. 2009. ``Information Needs.'' In
\emph{Encyclopedia of Library and Information Sciences, Third Edition},
null:2452--8. null. Taylor \& Francis.
\url{http://www.tandfonline.com/doi/abs/10.1081/E-ELIS3-120043243}.

\hypertarget{ref-saarinenux5fsignux5f2013}{}
Saarinen, Katariina, and Pertti Vakkari. 2013. ``A Sign of a Good Book:
Readers' Methods of Accessing Fiction in the Public Library.''
\emph{Journal of Documentation} 69 (5): 736--54.
doi:\href{https://doi.org/10.1108/JD-04-2012-0041}{10.1108/JD-04-2012-0041}.

\hypertarget{ref-sandstromux5fuseux5f1995}{}
Sandstrom, Alan R., and Pamela Effrein Sandstrom. 1995. ``The Use and
Misuse of Anthropological Methods in Library and Information Science
Research.'' \emph{The Library Quarterly: Information, Community, Policy}
65 (2): 161--99. \url{http://www.jstor.org/stable/4309020}.

\hypertarget{ref-shachafux5fareux5f2006}{}
Shachaf, Pnina, and Sarah Horowitz. 2006. ``Are Virtual Reference
Services Color Blind?'' \emph{Library \& Information Science Research}
28 (4): 501--20.
doi:\href{https://doi.org/10.1016/j.lisr.2006.08.009}{10.1016/j.lisr.2006.08.009}.

\hypertarget{ref-shiltonux5fparticipatoryux5f2007}{}
Shilton, Katie, and Ramesh Srinivasan. 2007. ``Participatory Appraisal
and Arrangement for Multicultural Archival Collections.''
\emph{Archivaria} 63: 87.
\url{http://rameshsrinivasan.org/wordpress/wp-content/uploads/2010/03/8-Final-ShiltonSrinivasan-Archivaria.pdf}.

\hypertarget{ref-swogerux5f.ux5f2012}{}
Swoger, B. 2012.`` What Is Metadata? A Christmas-Themed Exploration.
Info Culture.'' \emph{Sicientific American}.
\href{http://blogs.scientificamerican.com/information-\%20culture/2012/12/17/what-is-metadata-a-christmas-themed-exploration/}{http://blogs.scientificamerican.com/information- culture/2012/12/17/what-is-metadata-a-christmas-themed-exploration/}.

\hypertarget{ref-teperux5fselectionux5f2014}{}
Teper, Jennifer Hain. 2014. ``Selection for Preservation.''
\emph{Library Resources \& Technical Services} 58 (4): 220--32.
\url{http://search.ebscohost.com/login.aspx?direct=true\&db=a9h\&AN=99263271\&site=ehost-live}.

\hypertarget{ref-ux5fupdateux5f2013}{}
``Update on the Twitter Archive at the Library of Congress.'' 2013.
Library of Congress.
\url{http://www.loc.gov/today/pr/2013/files/twitter_report_2013jan.pdf}.

\hypertarget{ref-vogtux5fescienceux5f2013}{}
Vogt, Lars. 2013. ``eScience and the Need for Data Standards in the Life
Sciences: In Pursuit of Objectivity Rather Than Truth.''
\emph{Systematics and Biodiversity} 11 (3): 257--70.
doi:\href{https://doi.org/10.1080/14772000.2013.818588}{10.1080/14772000.2013.818588}.

\hypertarget{ref-petersux5fphilosophyux5f2002}{}
Voorhees, EllenM. 2002. ``The Philosophy of Information Retrieval
Evaluation.'' In \emph{Evaluation of Cross-Language Information
Retrieval Systems}, edited by Carol Peters, Martin Braschler, Julio
Gonzalo, and Michael Kluck, 2406:355--70. Lecture Notes in Computer
Science. Springer Berlin Heidelberg.
\url{http://dx.doi.org/10.1007/3-540-45691-0_34}.

\hypertarget{ref-warrenux5f2015ux5f2015}{}
Warren, John. 2015. ``(2015) Zen and the Art of Metadata Maintenance.''
\emph{Journal of Electronic Publishing} 18 (3).
\url{http://dx.doi.org/10.3998/3336451.0018.305}.

\hypertarget{ref-whitmireux5fracialux5f1999}{}
Whitmire, Ethelene. 1999. ``Racial Differences in the Academic Library
Experiences of Undergraduates.'' \emph{The Journal of Academic
Librarianship} 25 (1): 33--37.
doi:\href{https://doi.org/10.1016/S0099-1333(99)80173-6}{10.1016/S0099-1333(99)80173-6}.

\hypertarget{ref-wilsonux5finformationux5f2008}{}
Wilson, Tom. 2008. ``The Information User: Past, Present and Future.''
\emph{Journal of Information Science} 34 (4): 457--64.
doi:\href{https://doi.org/10.1177/0165551508091309}{10.1177/0165551508091309}.

\hypertarget{ref-wrightux5fcatalogingux5f2014}{}
Wright, Alex. 2014a. \emph{Cataloging the World : Paul Otlet and the
Birth of the Information Age}. Oxford: Oxford University Press.
\url{http://vufind.carli.illinois.edu/vf-uiu/Record/uiu_7507894}.

\hypertarget{ref-wrightux5fsecretux5f2014}{}
---------. 2014b. ``The Secret History of Hypertext --- The Atlantic.''
\url{http://www.theatlantic.com/technology/archive/2014/05/in-search-of-the-proto-memex/371385/}.

\end{document}
